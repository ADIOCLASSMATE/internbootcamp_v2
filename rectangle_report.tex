\documentclass[12pt, a4paper]{article}

% 导入中文支持宏包 (建议使用 xelatex 编译)
\usepackage{ctex}
\setmainfont{Times New Roman}

% 页面边距设置
\usepackage[top=2.5cm, bottom=2.5cm, left=2.5cm, right=2.5cm]{geometry}

% 图片和图形相关宏包
\usepackage{graphicx}
\usepackage{float}
\usepackage{subcaption}
\usepackage{booktabs}
\usepackage{listings}
\usepackage{xcolor}
\usepackage{amsmath}
\usepackage{newunicodechar}
\usepackage[most]{tcolorbox}

% 定义特殊字符
\newunicodechar{█}{\rule{1ex}{1ex}}

% 定义颜色
\definecolor{codegreen}{rgb}{0,0.6,0}
\definecolor{codegray}{rgb}{0.5,0.5,0.5}
\definecolor{codepurple}{rgb}{0.58,0,0.82}
\definecolor{backcolour}{rgb}{0.95,0.95,0.92}

% 代码块设置
\lstset{
    basicstyle=\ttfamily\small,
    breaklines=true,
    frame=single,
    backgroundcolor=\color{backcolour},
    keywordstyle=\color{magenta},
    commentstyle=\color{codegreen},
    stringstyle=\color{codepurple},
    numberstyle=\tiny\color{codegray},
    numbers=left,
    numbersep=5pt,
    showspaces=false,
    showstringspaces=false,
    showtabs=false,
    tabsize=2,
    captionpos=b
}

% 标题信息
\title{\textbf{基于 Rectangle Count 任务的大语言模型文本推理能力评测}}
\author{蒋维 \quad 万佳欣 \quad 晁磊玉}
\date{\today}

\begin{document}

\maketitle

\begin{abstract}
本报告针对纯文本推理中的 Rectangle Count(矩形计数)任务进行了深入研究。我们首先分析了任务特性,并设计了优化的 Prompt Template 以规范模型输出。随后,我们在 DeepSeek-R1 模型上进行了 Ablation Study,探究了 Grid Size 对模型性能的影响,并根据 Error Analysis 调整了 Reward Function,从 Linear Decay 改为更严格的 Binary Reward,以避免模型通过猜测获得高分。最后,我们在 Qwen-2.5 和 Qwen-3 系列模型上进行了全面的评测,采用了论文中的 Easy and Hard Difficulty Levels,验证了模型规模与性能之间的 Scaling Law。
\end{abstract}

\section{任务描述与评测流程}

\subsection{Rectangle Count 任务}
Rectangle Count 任务要求模型识别并计数 ASCII 网格中的矩形数量。单个矩形由 `\#` 字符勾勒,重叠矩形(最多 2 个)由 `█` 字符表示。该任务考察了模型的 Logical Reasoning Capability 和对复杂图形结构的理解能力。

\subsection{Prompt Optimization}
为了提高评测的稳定性,确保模型能够按照指定格式输出答案,我们设计了如下的 Prompt Template:

\begin{tcolorbox}[colback=gray!5!white, colframe=gray!75!black, title=Prompt Template, fonttitle=\bfseries\sffamily, arc=1mm, boxrule=0.5pt, breakable]
\begin{lstlisting}[basicstyle=\rmfamily\small, backgroundcolor={}, language={}, numbers=none, frame=none, breaklines=true, columns=fullflexible]
Your task is to count how many rectangles are present in an ASCII grid.

Single rectangles are outlined with a '#', overlapping rectangles (max 2) are shown with '█'.

Your output should be a single number, representing the total count of rectangles, and put it in the format \\boxed{{number}}.

Now, it's your turn. How many rectangles do you see in the grid below?
{puzzle}
\end{lstlisting}
\end{tcolorbox}

该模板明确了任务目标、符号含义以及输出格式($\backslash$boxed\{\{number\}\}),有助于后续的自动化评测与解析。

\section{实验一:DeepSeek-R1 Ablation Study 与 Error Analysis}

鉴于 reasoning-gym 论文详细报告了 DeepSeek-R1 模型在 Rectangle Count 任务上的性能表现——在标准的 $80 \times 80$ 网格设置下,其 Easy Setting 得分为 46.0,Hard Setting 得分为 16.0——我们选取该模型作为前期验证实验的 Base Model。这不仅有助于我们将复现结果与论文基准进行对齐,也为后续的 Ablation Study 和评测流程优化提供了可靠的参照系。

\begin{table}[htbp]
    \centering
    \caption{论文中 DeepSeek-R1 在 80$\times$80 网格下的基准性能}
    \label{tab:deepseek-baseline}
    \begin{tabular}{llc}
    \toprule
    \textbf{Grid Size} & \textbf{Difficulty} & \textbf{Score} \\
    \midrule
    80$\times$80 & Easy & 0.46 \\
                 & Hard & 0.16 \\
    \bottomrule
    \end{tabular}
\end{table}

\subsection{实验假设与设置}
我们假设在矩形数量相同的情况下,输入数据的 Grid Size 越大,矩形重叠的概率越低,任务相对越简单。为了验证这一假设,我们在 DeepSeek-R1 模型上进行了 Ablation Study,设置了不同的 Grid Size 配置,各配置生成100个测试样本,并记录模型的表现。

\subsection{实验结果}
实验结果如表 \ref{tab:deepseek-ablation} 所示。

\begin{table}[H]
    \centering
    \caption{DeepSeek-R1 模型在不同 Grid Size 下的 Ablation Study 结果}
    \label{tab:deepseek-ablation}
    \begin{tabular}{llcc}
\toprule
\textbf{Grid Size} & \textbf{Difficulty} & \textbf{Avg. Score} & \textbf{Avg. Tokens} \\
\midrule
20*20 & Easy & 0.3286 & 11232.39 \\
 & Hard & 0.2980 & 10766.52 \\
\addlinespace
40*40 & Easy & 0.4726 & 9664.04 \\
 & Hard & 0.3574 & 8764.76 \\
\addlinespace
60*60 & Easy & 0.5230 & 8439.20 \\
 & Hard & 0.4114 & 7760.34 \\
\addlinespace
80*80 & Easy & 0.5518 & 8176.77 \\
 & Hard & 0.4546 & 7473.17 \\
\bottomrule
\end{tabular}
\end{table}

结果显示,随着 Grid Size 的增加,模型的 Accuracy 呈现上升趋势,验证了我们的假设。较大的画布减少了图形的密集重叠,降低了解析的难度。我们的 Prompt 在相同的 Setting 下,帮助模型超过了论文中的基准分数。

\subsection{Error Analysis 与 Reward Function Adjustment}
在分析 DeepSeek-R1 的 Ablation Study 输出时,我们发现模型有时会产生类似如下的错误推理:

\begin{quote}
In ASCII art, if two rectangles intersect... However, these are line segments, not rectangles.
So it's probably not.
Given the difficulty, I'll assume the answer is 8.
I output \boxed{8}.
\end{quote}

在这种情况下,模型在不确定的情况下进行猜测。我们最初参考相关论文使用的 Reward Function 为 Linear Decay:
$$ R_{answer} = \begin{cases} 0.9 & \text{if } |y_{pred} - y_{true}| = 0 \\ 0.9 \times (1 - \frac{|y_{pred} - y_{true}|}{5}) & \text{if } 0 < |y_{pred} - y_{true}| < 5 \\ 0 & \text{if } |y_{pred} - y_{true}| \ge 5 \end{cases} $$

由于 Linear Decay 机制,猜测接近的答案也能获得较高的 Reward,这不能真实反映模型的 Reasoning Capability。因此,为了更严谨地评估模型性能,我们将 Answer Reward 修改为固定的 Binary Reward:
$$ R_{answer} = \begin{cases} 0.9 & \text{if } y_{pred} = y_{true} \\ 0.1 & \text{if } y_{pred} \neq y_{true} \end{cases} $$
答对得 0.9 分,答错得 0.1 分。
此外,若模型输出符合格式要求,额外给予 0.1 分的 Format Reward。

\section{实验二:DeepSeek 与 Qwen Series Models Evaluation}

在完成 Ablation Study 与 Error Analysis 后,我们正式对 DeepSeek 系列(DeepSeek-R1, DeepSeek-V3)与 Qwen 系列(Qwen2.5, Qwen3, QwQ)模型进行了大规模评测。在此阶段,我们采用了更严格的 Answer Reward Calculation:仅当预测答案与真实答案完全一致时得 0.9 分,否则得 0.1 分。这一调整旨在消除模型猜测带来的分数偏差,更真实地反映模型的精确 Reasoning Capability。

\subsection{Evaluation Results}
所有模型的评测结果汇总于表 \ref{tab:model-comparison}。

\begin{table}[htbp]
    \centering
    \begin{tabular}{lccc}
\toprule
\textbf{Model} & \textbf{Size} & \textbf{Easy Score} & \textbf{Hard Score} \\
\midrule
DeepSeek-V3 & - & 0.3520 & 0.2890 \\
DeepSeek-R1 & - & 0.3610 & 0.3250 \\
QwQ-32B & 32.0B & 0.3280 & 0.2900 \\
Qwen2.5-1.5B-Instruct & 1.5B & 0.1090 & 0.0900 \\
Qwen2.5-3B-Instruct & 3.0B & 0.1040 & 0.1020 \\
Qwen2.5-7B-Instruct & 7.0B & 0.1970 & 0.2020 \\
Qwen2.5-14B-Instruct & 14.0B & 0.3510 & 0.3320 \\
Qwen2.5-32B-Instruct & 32.0B & 0.3780 & 0.3510 \\
Qwen2.5-72B-Instruct & 72.0B & 0.4400 & 0.3850 \\
Qwen3-1.7B & 1.7B & 0.0550 & 0.0560 \\
Qwen3-4B & 4.0B & 0.1670 & 0.2080 \\
Qwen3-8B & 8.0B & 0.2020 & 0.1950 \\
Qwen3-14B & 14.0B & 0.2510 & 0.2490 \\
Qwen3-32B & 32.0B & 0.2810 & 0.2650 \\
Qwen3-235B-A22B-Instruct-2507 & 235.0B & 0.4850 & 0.3860 \\
\bottomrule
\end{tabular}

    \caption{DeepSeek 与 Qwen Series Models 在不同 Difficulty Levels 下的评测结果}
    \label{tab:model-comparison}
\end{table}

为了直观展示模型规模对性能的影响,我们展示 Qwen2.5 与 Qwen3 系列模型的评测结果折线图,如图 \ref{fig:qwen2.5-charts} 和图 \ref{fig:qwen3-charts} 所示。

\begin{figure}[H]
    \centering
    \includegraphics[width=0.8\textwidth]{rectangle/qwen2.5_performance_comparison.png}
    \caption{Qwen2.5 Series Models 在 Easy 和 Hard Setting 下的性能随参数量变化的趋势图}
    \label{fig:qwen2.5-charts}
\end{figure}

\begin{figure}[H]
    \centering
    \includegraphics[width=0.8\textwidth]{rectangle/qwen3_performance_comparison.png}
    \caption{Qwen3 Series Models 在 Easy 和 Hard Setting 下的性能随参数量变化的趋势图}
    \label{fig:qwen3-charts}
\end{figure}

从结果可以看出,DeepSeek-R1 在 Easy 和 Hard Setting 下均表现出色,但是由于采用了更加严格的answer reward计算方式,分数较论文中的指标有所下降。Qwen Series Models 整体呈现出 Scaling Law 的趋势,即随着参数量的增加,模型性能显著提升。特别是 Qwen3-235B-A22B-Instruct-2507 模型,在 Easy Setting 下取得了最高的 0.4850 分。
\section{Conclusion}
本次实验通过优化 Prompt 和 Reward Function,建立了一套更严谨的 Rectangle Count 任务评测流程。DeepSeek-R1 的 Ablation Study 揭示了输入数据的 Grid Size 对推理的影响,同时证明了我们优化后的prompt会获得比原论文中更高的分数。Qwen Series Models 的评测则验证了模型规模对任务性能的显著贡献。

\section{Model Output CoT Examples}

为了直观展示模型的推理过程,我们选取了 DeepSeek-R1 分别在 Easy Difficulty 与 Hard Difficulty 的输出示例。

\subsection{DeepSeek-R1 Output Example 1 (Easy Setting)}
\begin{tcolorbox}[colback=gray!5!white, colframe=gray!75!black, title=Input Prompt, fonttitle=\bfseries\sffamily, arc=1mm, boxrule=0.5pt, breakable]
\lstinputlisting[basicstyle=\ttfamily\small, backgroundcolor={}, numbers=none, frame=none, breaklines=true]{rectangle/deepseek_r1_prompt_1.txt}
\end{tcolorbox}

\begin{tcolorbox}[colback=gray!5!white, colframe=gray!75!black, title=Model Response, fonttitle=\bfseries\sffamily, arc=1mm, boxrule=0.5pt, breakable]
\lstinputlisting[basicstyle=\ttfamily\small, backgroundcolor={}, numbers=none, frame=none, breaklines=true, columns=fullflexible]{rectangle/deepseek_r1_response_1.txt}
\end{tcolorbox}

\subsection{DeepSeek-R1 Output Example 2 (Hard Setting)}
\begin{tcolorbox}[colback=gray!5!white, colframe=gray!75!black, title=Input Prompt, fonttitle=\bfseries\sffamily, arc=1mm, boxrule=0.5pt, breakable]
\lstinputlisting[basicstyle=\ttfamily\small, backgroundcolor={}, numbers=none, frame=none, breaklines=true]{rectangle/deepseek_r1_prompt_2.txt}
\end{tcolorbox}

\begin{tcolorbox}[colback=gray!5!white, colframe=gray!75!black, title=Model Response, fonttitle=\bfseries\sffamily, arc=1mm, boxrule=0.5pt, breakable]
\lstinputlisting[basicstyle=\ttfamily\small, backgroundcolor={}, numbers=none, frame=none, breaklines=true, columns=fullflexible]{rectangle/deepseek_r1_response_2.txt}
\end{tcolorbox}

\end{document}